\section{Discussion}
\label{sec:discussion}

\subsection{Summary of Empirical Findings}

Our multi-scale analysis across eight Pythia models (70M to 12B parameters) reveals three universal principles and one scale-dependent phenomenon:

\textbf{Universal Principles:}
\begin{enumerate}
    \item \textbf{Attention is invariably contractive.} Across all models, 97--100\% of attention layers exhibit gain $< 1$, compressing representations toward a harmonic subspace.
    \item \textbf{The final MLP layer always expands.} Every model tested shows final layer MLP gain $> 1$, with values ranging from 1.5$\times$ (70M) to 7.7$\times$ (12B).
    \item \textbf{Net compression dominates.} Despite varying MLP behavior, 94--96\% of layers show net contraction (attention $\times$ MLP gain $< 1$).
\end{enumerate}

\textbf{Scale-Dependent Phenomenon:}
\begin{enumerate}
    \setcounter{enumi}{3}
    \item \textbf{MLP contraction decreases with scale.} Small models show 75--88\% MLP-contracting layers, while Pythia-12B shows only 6\%---a dramatic shift from compression to expansion.
\end{enumerate}

These findings culminate in a robust scaling law:
\begin{equation}
    \text{Final\_MLP\_Gain} = 0.013 \times \text{Params}^{0.265 \pm 0.079}
\end{equation}
with $R^2 = 0.65$ and $p = 0.015$, confirming statistical significance.

\subsection{The Cognitive Inertia Hypothesis}

We propose that the observed scaling law reflects a fundamental property of deep attention networks that we term \textbf{cognitive inertia}: the tendency of larger models to resist prediction in favor of representational stability.

\subsubsection{Physical Analogy}

Consider Newton's second law: $F = ma$. In the context of Transformers:

\begin{table}[h]
\centering
\begin{tabular}{ll}
\toprule
Physical Quantity & LLM Analog \\
\midrule
Mass ($m$) & Contextual stability ($\propto$ parameters, depth) \\
Force ($F$) & Final MLP expansion gain \\
Acceleration ($a$) & Decision sharpness (logit spread) \\
\bottomrule
\end{tabular}
\end{table}

Our scaling law states $F \propto m^{0.265}$, implying that the ``force'' required to produce a decision grows with model ``mass''---but sublinearly, suggesting a cumulative lever effect across layers.

\subsubsection{The Stability-Decision Tradeoff}

Large models build increasingly stable representations through repeated attention operations. Each attention layer acts as a diffusion operator, smoothing representations toward consensus. This creates what we call a \textbf{representational potential well}---a stable attractor state that resists perturbation.

However, next-token prediction fundamentally requires \textit{breaking} this consensus. The model must commit to a single token from a vocabulary of 50,000+, necessitating sharp discrimination rather than smooth consensus.

The final MLP layer serves as the \textbf{symmetry-breaking mechanism}. Its expansion gain represents the energy required to escape the potential well created by preceding layers. Larger models create deeper wells, requiring larger explosions.

\subsection{Thermodynamic Interpretation}

\subsubsection{Entropy Flow Through Layers}

\begin{table}[h]
\centering
\begin{tabular}{llll}
\toprule
Phase & Layers & Operation & Entropy \\
\midrule
Consensus & $0$ to $L-1$ & Attention averaging & $\downarrow$ Decreasing \\
Bottleneck & $L^*$ & Maximum compression & Minimum \\
Explosion & $L$ & MLP expansion & $\uparrow$ Increasing \\
\bottomrule
\end{tabular}
\end{table}

The attention mechanism acts as a \textbf{cooling process}, ordering information into a low-entropy harmonic subspace. The final MLP acts as a \textbf{heating process}, injecting entropy to enable discrimination.

\subsubsection{The Second Law for LLMs}

We conjecture a principle analogous to the second law of thermodynamics:

\begin{quote}
\textit{The work required to break representational equilibrium grows with the depth of that equilibrium.}
\end{quote}

Formally, if $\mathcal{S}_{\text{harm}}$ denotes the harmonic subspace stability and $W_{\text{decision}}$ denotes the work for decision-making:
\begin{equation}
    W_{\text{decision}} \propto \|\mathcal{S}_{\text{harm}}\|^{\alpha}
\end{equation}
Our measured $\alpha \approx 0.27$ quantifies this relationship empirically.

\subsection{Architectural Critique: The Consensus-Discrimination Paradox}

Our findings reveal a fundamental tension in Transformer design:

\begin{table}[h]
\centering
\begin{tabular}{lll}
\toprule
Design Goal & Mechanism & Effect \\
\midrule
Context understanding & Attention (consensus) & Stabilizes representations \\
Token prediction & MLP expansion (discrimination) & Destabilizes representations \\
\bottomrule
\end{tabular}
\end{table}

\textbf{The paradox:} Transformers are optimized for consensus-building (attention), but their task requires discrimination (prediction). As models scale, attention becomes more effective at consensus---creating stable representations increasingly difficult to break.

\subsubsection{The VASE Architecture as Symptom}

Pythia-12B exhibits what we call the \textbf{VASE architecture}: wide throughout with minimal compression, culminating in massive final expansion (7.7$\times$). Only 6\% of MLP layers contract.

This represents \textit{architectural resignation}: the model has learned that compression is futile at its scale, because the final explosion dominates. It maintains representational breadth throughout, postponing all discrimination to the final layer.

\subsubsection{Evolution of Architecture with Scale}

We identify three architectural regimes:

\begin{itemize}
    \item \textbf{FUNNEL} (70M--410M): Wide $\rightarrow$ Narrow $\rightarrow$ Slightly Wide
    \item \textbf{HOUR-GLASS} (1B--2.8B): Wide $\rightarrow$ Narrow $\rightarrow$ Wide
    \item \textbf{VASE} (6.9B--12B): Wide $\rightarrow$ Wide $\rightarrow$ Very Wide
\end{itemize}

The evolution from FUNNEL to VASE suggests compression becomes relatively less important as models scale, while final expansion becomes relatively more important.

\subsection{The Scaling Paradox: Why Bigger May Hit Limits}

\subsubsection{Extrapolation to Frontier Models}

Applying our scaling law to larger models:

\begin{table}[h]
\centering
\begin{tabular}{lrr}
\toprule
Model & Parameters & Predicted Final MLP Gain \\
\midrule
Pythia-12B & 12B & 7.7$\times$ (measured) \\
LLaMA-70B & 70B & $\sim$13$\times$ \\
GPT-3 & 175B & $\sim$18$\times$ \\
GPT-4 (est.) & 1.8T & $\sim$30$\times$ \\
\bottomrule
\end{tabular}
\end{table}

\subsubsection{Potential Failure Modes}

As gain increases, several failure modes become likely:

\begin{enumerate}
    \item \textbf{Numerical Instability:} Gains of 30--50$\times$ risk gradient explosion and inference instability.
    \item \textbf{Overconfident Predictions:} Extreme expansion creates sharp, poorly-calibrated logit distributions.
    \item \textbf{Brittleness:} High-gain final layers amplify small perturbations.
    \item \textbf{Training Difficulty:} The gradient landscape becomes ill-conditioned.
\end{enumerate}

\subsubsection{The Hamlet Problem of AI}

We observe a philosophical dimension: larger models exhibit what we term the \textbf{Hamlet Problem}---the more the model knows, the harder it finds to act.

The extensive plateau phase (layers 7--30 in large models) represents prolonged ``deliberation'' before the explosive ``decision'' in the final layer. This mirrors the cognitive pattern of analysis paralysis.

\subsection{Implications for Future Architectures}

\subsubsection{Decoupling Consensus and Discrimination}

Future designs might:
\begin{itemize}
    \item \textbf{Parallel pathways:} Separate ``consensus'' and ``discrimination'' streams.
    \item \textbf{Gated discrimination:} Bypass consensus layers when confidence is high.
    \item \textbf{Sparse attention:} Maintain ``escape routes'' in attention patterns.
\end{itemize}

\subsubsection{Distributed Prediction}

Rather than concentrating prediction in the final layer:
\begin{itemize}
    \item \textbf{Early exits:} Predictions at multiple layers.
    \item \textbf{Mixture of depths:} Route tokens through variable layer counts.
\end{itemize}

\subsubsection{Non-Diffusive Attention}

Standard attention is inherently diffusive. Alternatives might include:
\begin{itemize}
    \item \textbf{Sharpening attention:} Mechanisms that sharpen rather than smooth.
    \item \textbf{Competitive attention:} Winner-take-all dynamics.
\end{itemize}

\subsection{Connection to Sheaf Theory}

Our findings provide empirical grounding for the sheaf-theoretic framework.

The universal contractivity of attention confirms that restriction maps $\rho_{ij} = \sqrt{A_{ij}} \cdot W_V$ are predominantly contractive, driving representations toward the harmonic subspace $\mathcal{H} = \ker(\Delta_{\mathcal{F}})$.

The final layer explosion can be interpreted as overcoming a \textbf{cohomological obstruction}. The harmonic subspace $\mathcal{H}$ represents sections satisfying the gluing axiom. Prediction requires leaving $\mathcal{H}$, obstructed by $H^1(G; \mathcal{F}) \neq 0$. The MLP expansion provides energy to overcome this obstruction.

The scaling exponent $\alpha \approx 0.27$ may reflect the rate at which cohomological complexity grows with model size---larger models have richer sheaf structures creating higher-dimensional obstruction spaces.

\subsection{Limitations}

\begin{itemize}
    \item \textbf{Model Family:} Analysis focuses on Pythia; generalization to other families requires validation.
    \item \textbf{Single Prompt:} All measurements use one prompt; different prompts may exhibit different dynamics.
    \item \textbf{Gain vs. Information:} Norm-based gain captures magnitude but not information content.
    \item \textbf{Causal Claims:} Our analysis is observational, identifying correlations not causation.
\end{itemize}

\subsection{Conclusion}

The scaling law $\text{Final\_MLP\_Gain} \propto \text{Params}^{0.27}$ reveals a fundamental tension in Transformer architectures. As models scale, they become increasingly effective at consensus, creating deep representational potential wells. Breaking these wells requires increasingly violent explosions.

This \textbf{cognitive inertia} suggests current scaling approaches may face diminishing returns. The thermodynamic cost of decision-making grows with capacity, potentially creating an efficiency barrier.

We term this the \textbf{Scaling Paradox}: bigger models are better at understanding, but increasingly struggle with deciding. Resolving this paradox may be key to next-generation architectures.
